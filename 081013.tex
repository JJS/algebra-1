\fdatum{13.10.2008}
\section{Grundlegende Definitionen}

\begin{defi}
Eine \textbf{Gruppe} ist eine Menge $G$ mit einer Abbildung $G\times G\rightarrow G$, $(g,h)\mapsto gh$ (\textit{innere Verkn�pfung}) derart, dass
\begin{enumerate}[(G1)]
\item Assoziativgesetz:
\[\forall g,h,j\in G:\qquad(gh)j=g(hj)\]
\item Existenz eines linksneutralen Elements:
\[\exists e\in G\:\forall g\in G:\qquad eg=g\]
\item Existenz von Linksinversen:
\[\forall g\in G\:\exists g^*\in G:\qquad g^*g=e\]
\end{enumerate}
Ist $gh=hg$ ($\forall g,h\in G$) so hei�t $G$ \textit{kommutativ} (\textit{abelsch}).\\[1ex]
$\vert G\vert\in\mathbb N\cup\{\infty\}$ ist die Anzahl der Elemente von $G$ (\textit{Ordnung von $G$})
\end{defi}

\begin{bem}[Halbgruppe und Monoid]
\noindent Gilt f�r  eine Menge $G$ nur das Assoziativgesetz \textit{(G1)}, handelt es sich um eine \textbf{Halbgruppe}.\\[1ex]
Existiert zus�tzlich ein neutrales Element $e\in G$ mit $eg=ge=g$ ($\forall g\in g$), spricht man von einem \textbf{Monoid}.
\end{bem}

\begin{bem}[Konsequenzen aus den Gruppenaxiomen]
\begin{enumerate}[(1)]
\item Das Linksneutrale in \textit{(G2)} ist eindeutig.
\item Das $g^*$ zu $g$ in \textit{(G3)} ist eindeutig.
\item Das Linksneutrale $e$ in \textit{(G2)} ist auch rechtsneutral, dh.
\[ge=g\qquad(\forall g\in G)\]
$e$ ist das neutrale Element der Gruppe.
\item Das Linksinverse $g^*$ zu $g$ ist auch rechtsinvers, dh.
\[gg^*=e\qquad(\forall g\in G)\]
$g^*$ ist das Inverse zu $g$ ($g^{-1}$ statt $g^*$)
\end{enumerate}
\end{bem}

\begin{bew}
\begin{enumerate}
\item[(4)] $g\in G$ gegeben. $g^*g=e$ und $g^{**}g^*=e$ nach \textit{(G3)}. Dann
\[gg^*=e(gg^*)=(g^{**}g^*)(gg^*)=g^{**}(eg^*)=g^{**}g^*=e\]
\item[(3)]
\[ge=g(g^*g)=(gg^*)g\overset{(4)}{=}eg=g\]
\item[(1)] Sei auch $e'\in G$ mit $e'g=g$ ($\forall g\in G$)\\[1ex]
$\Rightarrow$ $e'e=e$ und $e'e=e'$ nach \textit{(3)}\\[1ex]
$\Rightarrow$ $e=e'$
\item[(2)] Sei neben $g^*g=e$ auch $g'g=e$:
\[g'=g'e=g'(g^*g)=g'(gg^*)=(g'g)g^*=eg^*=g^*\]
\end{enumerate}
\end{bew}

\begin{bei}[$(\mathbb Z,+)$, $(a,b)\mapsto a+b$]
\begin{itemize}
\item Neutrales Element: $0$
\item Inverse zu $a\in\mathbb Z$: $(-a)$ denn $(-a)+a=0$
\end{itemize}
$(\mathbb Z,+)$ ist kommutative Gruppe
\end{bei}

\begin{bem}[Rechenregeln in Gruppen]
\begin{enumerate}[(1)]
\item
\[\left(g^{-1}\right)^{-1}=g\qquad(\forall g\in G)\]
\item
\[(gh)^{-1}=h^{-1}g^{-1}\qquad(\forall g,h\in G)\]
\end{enumerate}
\end{bem}

\begin{bew}
\begin{enumerate}[(1)]
\item $g^{-1}g=e$: Das Inverse von $g^{-1}$ ist $g$:
\[\left(g^{-1}\right)^{-1}=g\]
\item $\left(h^{-1}g^{-1}\right)(gh)=e$: Das Inverse von $(gh)$ ist $h^{-1}g^{-1}$:
\[(gh)^{-1}=h^{-1}g^{-1}\]
\end{enumerate}
\end{bew}

\begin{defi}[Untergruppen]
$G$ Gruppe. Eine Teilmenge $H\subset G$ hei�t \textbf{Untergruppe} von $G$, wenn $H$ mit der Verkn�pfung von $G$ selbst eine Gruppe ist.
\end{defi}

\begin{bei}
\[(2\mathbb Z,+)\underset{\text{Untergruppe}}{\leq}(\mathbb Z,+)\]
\end{bei}

\begin{bem}
Sei $H\leq G$ und sei $e_H$ das neutrale Element von $H$ und $e_G$ das neutrale Element von $G$. Dann ist
\[e_H=e_G\]
Sei $h\in H$ und $h'$ das Inverse von $h$ in $H$ und $h^{-1}$ das Inverse von $h$ in $G$. Dann ist
\[h'=h^{-1}\]
\end{bem}

\begin{bem}[Untergruppenkriterium]
$G$ Gruppe und $H\leq G$.
\[H\leq G\:\Leftrightarrow\:H\neq\emptyset\wedge\left(g,h\in H\:\Rightarrow\:gh^{-1}\in H\right)\]
\end{bem}

\begin{bew}
\begin{enumerate}
\item[$\Rightarrow$] klar
\item[$\Leftarrow$]
\begin{itemize}
\item \[(1):\:\exists h\in H\:\overset{(2)}{\Rightarrow}\:hh^{-1}\in H\:\Rightarrow\:e\in H\]
\item F�r $h\in H$ ist auch $eh^{-1}=h^{-1}\in H$
\item $g,h\in H$ $\Rightarrow$ $g,h^{-1}\in H$ $\overset{(2)}{\Rightarrow}$ $g\left(h^{-1}\right)^{-1}\in H$, dh. $gh\in H$
\end{itemize}
\end{enumerate}
\end{bew}

\begin{defi}
Ein \textbf{Ring} ist eine Menge $R$ mit zwei Abbildungen (\textit{innere Verkn�pfungen})
\begin{itemize}
\item $R\times R\rightarrow R$, $(a,b)\mapsto a+b$ (Addition auf $R$)
\item $R\times R\rightarrow R$, $(a,b)\mapsto ab$ (Multiplikation auf $R$)
\end{itemize}
derart, dass gilt:
\begin{enumerate}[(R1)]
\item $(R,+)$ ist abelsche Gruppe (neutrales Element $0$, $0_R$)
\item $(R,\cdot)$ ist Monoid (neutrales Element $1$, $1_R$)
\item Die Distributivgesetze: $\forall a,b,c\in R$:
\begin{itemize}
\item $a(b+c)=ab+ac$
\item $(a+b)c=ac+bc$
\end{itemize}
\end{enumerate}
$R$ hei�t \textit{kommutativ} wenn die Multiplikation kommutativ ist. (Ring = Ring mit Eins)
\end{defi}

\begin{bei}
\begin{itemize}
\item $(\mathbb Z,+,\cdot)$
\item $M_{n\times n}(\mathbb R)$ f�r $n\geq2$ nicht kommutativ
\item $(2\mathbb Z,+,\cdot)$ kein Ring
\item $R=\{O\}$ mit $O+O:=O$ und $O\cdot O:=0$ ($1_R=0_R$)
\item $\vert R\vert>1$ $\Rightarrow$ $1_R\neq0_R$
\end{itemize}
\end{bei}