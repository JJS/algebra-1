\fdatum{17.10.2008}
\begin{bem}
r \textbf{Einheit} von $R$ $\Leftrightarrow$ $\exists s\in R\::\:rs=sr=1_R$\\[1ex]
$E(R)=R^X=$ Menge aller Einheiten von $R$, ist Gruppe bez�glich $\cdot$.
\end{bem}

\begin{bei}
\begin{itemize}
\item $E(\mathbb Z)=\{\pm1\}$
\item $E(M_{n\times n}(\mathbb R))=\{A\in M_{n\times n}(\mathbb R)\::\:A\neq0\}=GL_n(\mathbb R)$
\item Ist $R$ kommutativer Ring mit $R^X=R\setminus\{0\}$, so hei�t $R$ \textbf{K�rper} (z.B. $\mathbb Q$, $\mathbb R$, $\mathbb C$)
\end{itemize}
\end{bei}

\newpage

\section{Die ganzen Zahlen}

\begin{satz}[Division mit Rest]
Zu $a,b\in\mathbb Z$, $b>0$ existiert genau ein $q\in\mathbb Z$ mit
\[0\leq a-q\cdot b<b\]
\end{satz}

\begin{bew}
\[a+b\mathbb Z=\{a+bk\::\:k\in\mathbb Z\}\]
enth�lt Zahlen aus $\mathbb N_0$, zB. $a+b\vert a\vert$\\[1ex]
Induktionsaxiom (Jede nichtleere Teilmenge von $\mathbb N_0$ enth�lt ein kleinstes Element)
\[r:=\min\left(a+b\mathbb Z\right)\cap\mathbb N_0\]
Dann: $r\geq0$ und $r=a-q\cdot b$\\[1ex]
W�re $r\geq b$, so w�re
\[0\geq r':=r-b=a-(q+1)b\in a+b\mathbb Z\]
und $r'<r$. Widerspruch zur Wahl von $r$.\\[1ex]
Eindeutigkeit von $q$:
\[0\leq a-qb<b\]
Es sei auch
\[a\leq a-q'b<b\:\Rightarrow\:-b<-a+qb\leq0\]
Also
\[-b<(q'-q)b<b\]
und damit
\[-1<\underbrace{q'-q}_{\in\mathbb Z}<1\]
$\Rightarrow$ $q=q'$
\end{bew}

\begin{satz}[Die Untergruppen von $(\mathbb Z,+)$]
\begin{enumerate}[1)]
\item F�r jedes $n\in\mathbb N_0$ ist $n\mathbb Z$ Untergruppe von $\mathbb Z$
\item Ist $M\leq\mathbb Z$, so existiert genau ein $n\in\mathbb N_0$ mit $M=n\mathbb Z$
\end{enumerate}
\end{satz}

\begin{bew}
\begin{enumerate}[1)]
\item mit dem Untergruppenkriterium: $n\mathbb Z\neq\emptyset$ und
\[a,b\in n\mathbb Z\:\Rightarrow\:\exists a',b'\in\mathbb Z\::\:a=na',\:b=nb'\:\Rightarrow\:a-b=n(a'-b')\in n\mathbb Z\]
\item $M=\{0\}$. Dann eindeutig $M=0\mathbb Z$\\[1ex]
Sei also $M\neq\{0\}$. Dann existiert $c\in M\setminus\{0\}$ $\Rightarrow$ $-c\in M$, weil $M$ Gruppe.\\[1ex]
$\Rightarrow$ $M\cap\mathbb N\neq\emptyset$.\\[1ex]
Induktionsaxiom: $n:=\min M\cap\mathbb N$.\\[1ex]
Behauptung: $M=n\mathbb Z$
\begin{enumerate}
\item[$\supset$] $n\in M$ $\underset{\Rightarrow}{M\text{ Gruppe }}$ $\mathbb N\subset M$ $\underset{\Rightarrow}{M\text{ Gruppe}}$ $n\mathbb Z\subset M$
\item[$\subset$] Sei $a\in M$. Division von $a$ durch $n$:
\[a\leq a-qn<n\]
f�r passendes $q\in\mathbb Z$.
\[r=a-qn\in M\]
denn $a\in M$, $n\in M$, $M$ Gruppe.\\[1ex]
Also: $r\in\mathbb N_0\cap M$. Wegen $r<n$ folgt nach Wahl von $n$:
\[r=0\]
$\Rightarrow$ $a=qn\in n\mathbb Z$\\[1ex]
Eindeutigkeit: Seien $n_1,n_2\in\mathbb N_0$ mit
\[n_1\mathbb Z=b_2\mathbb Z\]
\[n_1=0\:\Leftrightarrow\:n_2=0\]
Sei also $n_1\neq0\neq n_2$.
\[\min\left(n_1\mathbb Z\cap\mathbb N\right)=\min\left(n_2\mathbb Z\cap\mathbb N\right)\]
\end{enumerate}
\end{enumerate}
\end{bew}

\begin{bem}[Teilbarkeit von $\mathbb Z$]
\[t,a\in\mathbb Z\::\:t\vert a\:(t\text{ teile }a)\:\Leftrightarrow\:\exists b\in\mathbb Z\::tb=a\]
\end{bem}

\begin{bei}
\begin{itemize}
\item $0\vert a$ $\Leftrightarrow$ $a=0$
\item $a\vert0$ gilt stets
\end{itemize}
Es gilt: $t\vert a$ $\Leftrightarrow$ $t\mathbb Z\supset a\mathbb Z$.
\end{bei}

\begin{defi}[Der gr��te gemeinsame Teiler]
Zu $a_1,\dots,a_n\in\mathbb Z$ ist
\[\sum_{i=1}^na_i\mathbb Z=\left\{\sum_{i=1}^na_ib_i\::\:b_i\in\mathbb Z\:(1\leq i\leq n)\right\}\]
ist Untergruppe von $\mathbb Z$, nach \textit{Satz 2(2)} existiert also ein $d\in\mathbb N_0$ mit
\[\sum_{i=1}^na_i\mathbb Z=d\mathbb Z\qquad(\star)\]
$d$ hei�t der gr��te gemeinsame Teiler der $a_i$
\[d=ggT(a_1,\dots,a_n)=gcd(a_1,\dots,a_n)\]
Insbesondere $d=\sum_{i=1}^na_ib_i$ mit passenden $b_i\in\mathbb Z$.
\end{defi}

\begin{bem}[Drei charakteristische Eigenschaften f�r $d=ggT(a_1,\dots,a_n)$]
\begin{enumerate}[(1)]
\item $d\in\mathbb N_0$
\item $d\vert a_i$ ($1\leq i\leq n$)
\item Ist $t\in\mathbb Z$ mit $t\vert a_i$ ($1\leq i\leq n$), so folgt $t\vert d$.
\end{enumerate}
\end{bem}

\begin{bew}
\begin{enumerate}[(1)]
\item klar
\item $(\star)$ $\Rightarrow$ $a_i\mathbb Z\subset d\mathbb Z$ $\Rightarrow$ $d\vert a_i$
\item $t\vert a_i$ $\Rightarrow$ $t\mathbb Z\supset a_i\mathbb Z$ ($1\leq i\leq n$) $\Rightarrow$ $t\mathbb Z\supset\sum_{i=1}^na_i\mathbb Z=d\mathbb Z$ (weil $t\mathbb Z$ Gruppe) $\Rightarrow$ $t\vert d$
\end{enumerate}
Umgekehrt:\\[1ex]
Erf�llt $d'\in\mathbb Z$ \textit{(1)-(3)}, so folgt $d'=d$\\[1ex]
zz.: $d'\vert d$, $d\vert d$ $\Rightarrow$ $d=d'$\\[1ex]
$d'\mathbb Z\supset d\mathbb Z$ und $d\mathbb Z\supset d'\mathbb Z$
\[d'\mathbb Z=d\mathbb Z,\:d,d'\in\mathbb N_0\:\underset{\Rightarrow}{\text{\textit{Satz 2}}}\:d=d'\]
\end{bew}

\begin{satz}[Rechenregeln f�r den ggT]
\begin{enumerate}[(1)]
\item $ggT(a,b)=ggT(b,a-qb)$ ($\forall q\in\mathbb Z$)
\item $ggT(a_1,\dots,a_n)=ggT(a_1,ggT(a_2,\dots,a_n))$ ($\forall n\geq3$)
\item $ggT(ba_1,\dots,ba_n)=\vert b\vert ggT(a_1,\dots,a_n)$
\item $ggT(a,bc)=ggT(a,b)$, falls $ggT(a,c)=1$ ist
\end{enumerate}
\end{satz}

\begin{bew}
\begin{enumerate}[(1)]
\item zz.: $\underbrace{a\mathbb Z+b\mathbb Z}_{ggT(a,b)\mathbb Z}=\underbrace{b\mathbb Z+(a-qb)\mathbb Z}_{ggT(b,a-qb)\mathbb Z}$.
\item
\[\sum_{i=1}^na_i\mathbb Z=a_1\mathbb Z+\left(\sum_{i=2}^na_i\mathbb Z\right)\]
\item
\[\sum_{i=1}^n(ba_i)\mathbb Z=\vert b\vert\sum_{i=1}^na_i\mathbb Z\]
\item zz.: $ggT(a,b)\vert ggT(a,bc)$ und $ggT(a,bc)\vert ggT(a,b)$
\begin{enumerate}[1. Teilbarkeit]
\item Benutzt werden die 3 charakteristischen Eigenschaften des ggT:
\begin{itemize}
\item $ggT(a,b)\vert a$
\item $ggT(a,b)\vert b$ $\Rightarrow$ $ggT(a,b)\vert bc$
\end{itemize}
$\underset{(3)}{\Rightarrow}$ $ggT(a,b)\vert ggT(a,bc)$
\item 
\begin{itemize}
\item $ggT(a,bc)\vert a$ $\Rightarrow$ $ggT(a,bc)\vert ab$
\item $ggT(a,bc)\vert bc$
\end{itemize}
$\underset{(3)}{\Rightarrow}$ $ggT(a,bc)\vert ggT(ab,bc)=\vert b\vert\underbrace{ggT(a,c)}_{=1}$ $\Rightarrow$ $ggT(a,bc)\vert b$ $\wedge$ $ggT(a,bc)\vert a$ $\Rightarrow$ $ggT(a,bc)\vert ggT(a,b)$
\end{enumerate}
\end{enumerate}
\end{bew}

\begin{bem}
\begin{enumerate}[(1)]
\item Ist $ggT(a_1,\dots,a_n)=1$, so hei�en $a_1,\dots,a_n$ teilerfremd (coprim).
\item Mit \textit{(2)} des Satzes l�sst sich die Berechnung von $ggT(a_1,\dots,a_n)$, $n\geq3$ reduzieren auf die Berechnung von ggT's zweier Zahlen. Daf�r gibt es einen schnellen Algorithmus: Wiederholte Division mit Rest:\\[1ex]
Gegeben: $a_0>a_1>0$\\[1ex]
Satz 1: $a_0=q_1a_1+a_2$ mit $0\leq a_2<a_1$\\[1ex]
Falls $a_2\leq0$: $a_1=q_2a_2+a_3$ mit $0\leq a_3<a_2$\\[1ex]
Die Restefolge $a_2,a_3,\dots$ ist strikt monoton fallend in $\mathbb N_0$, bricht also nach endlich vielen Schritten ab mit 0.
\begin{align*}
&\vdots\\
a_{n-1}&=q_na_n+a_{n+1}\text{ mit }a_{n+1}=0,\:a_n>0
\end{align*}
Dann ist $a_n=ggT(a_0,a_1)$, denn \textit{Satz 3(1)}:
\begin{align*}
ggT(a_0,a_1)&\underset{\text{1. Gleichung}}{=}ggT(a_1,a_2)\\
&\underset{\text{2. Gleichung}}{=}ggT(a_2,a_3)\\
&=ggT(a_n,a_{n+1})=ggT(a_n,0)=a_n
\end{align*}
\end{enumerate}
\end{bem}